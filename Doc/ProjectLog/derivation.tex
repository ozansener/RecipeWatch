\documentclass{article}
\usepackage{amsfonts}
\usepackage{amsmath}
\usepackage{bm}
% numbers option provides compact numerical references in the text.
\usepackage[numbers]{natbib}
\usepackage{multicol}
\usepackage[bookmarks=true]{hyperref}
\usepackage{graphicx}
\usepackage{color}
\usepackage[it,small]{caption}
\usepackage{subcaption}
\usepackage{dsfont}
\newcommand{\argmax}{\operatorname{arg\,max}}
\newcommand{\argmin}{\operatorname{arg\,min}}

\newcommand{\todo}[1]{\textcolor{blue}{\textbf{#1}}}

\newtheorem{example}{\hspace*{-1em}\textbf{Example}}

\title{Unsupervised Learning of Recipes from Large-Scale Multi-Modal Data}

\begin{document}
\maketitle
\section{Derivation of Non-Parametric Bayesian Model}
We consider large collection of videos and text recipes in order to learn recipes. In the initial setup, we consider the problem of jointly learning a single recipe from multiple videos and text inputs. We consider the following setup;
\begin{itemize}
  \item $N$ videos as $V_1,\ldots,V_N \quad s.t. \quad  V_k^i$ is $i^{th}$ frame of $k^{th}$ video (with subtitle)
  \item $M$ text recipes as $L_1,\ldots,L_M \quad  s.t. \quad  L_k^i$ is $i^{th}$ step of $k^{th}$ recipe
\end{itemize}
Moreover, we formulate our system on set of quantized visual objects and salient words. We consider;
\begin{itemize}
  \item $v_N$ visual objects as $w^v_1,\ldots,w^v_{v_N}$
  \item $l_N$ salient words as $w^l_1,\ldots,w^l_{l_N}$
\end{itemize}
As a latent concept, we consider;
\begin{itemize}
  \item $K$ Actions as $\theta_1,\ldots,\theta_K \quad s.t. \quad \theta_i \in \left(\underbrace{\{0,1\}}_{occ.}\times LM\right)^{v_N} \times [0,1]^{l_N} \times \underbrace{R}_{av.length}$
  \item Feature matrix $F \quad s.t. \quad$ $F_{i,j}=1$ if $i^{th}$ video has the $j^{th}$ activity, $0 \quad o.w.$
  \item Replecability Matrix $\quad R\quad s.t. \quad R_{i,j}=1$ if action $i$ and $j$ can be replaced
  \item Order Matrix $\quad O\quad s.t. \quad O_{i,j}=1$ if action $i$ can not occure before $j$
\end{itemize}
where $LM$ is \emph{LocalMotionSpace}. Local motion is represented as multinomial distribution over the clusters of optical flow histograms. We further know that $R$ is a symetric matrix and $O_{i,j}=0$ if $O_{j,i}=1$.

As a local latent concept per video, we have:
\begin{itemize}
  \item Local transition probabilities $\pi$ between activites.
  \item Selected features $f_i$
  \item Prior over transitions $\eta_i$
  \item Note: $\pi|\eta_i,f_i,RO \sim Dir\left(RO \cdot(f_i \bigodot \eta_i)\right)$
\end{itemize}

Moreover, as a modelling assumption
\begin{itemize}
  \item $F$ follows a beta process as $BP(B_0,c)$
  \item Videos follow HMM or CTMC
\end{itemize}

\begin{figure}
  \includegraphics[width=\textwidth]{plate}
\caption{Graphical Model of the Recipe Model}
\end{figure}

\section{Generative Description of the Model}
\begin{enumerate}
  \item We start with sampling Beta process realization $B\sim BP(c,B_0)$:
  \begin{equation}
    B = \sum_{k=1}^\infty w_k \theta_k \text{ where } \theta_k \sim B(w)^{v_N} \times Dir(\alpha)^{v_N} \times B(w)^{l_N} \times \Gamma(\cdot)
  \end{equation}
  \item We then sample co-not-occurance matrix through:
  \begin{equation}
    S^R = \Theta^T R \Theta  \rightarrow R = f(S) \bigodot M, M \sim  BeP(\cdot)
  \end{equation}
  \item We then sample ordering matrix similarly.
  \item Then for each video;
  \begin{itemize}
    \item Draw $f_i$ from $B$
    \item Draw transition functions $\pi_i \sim Dir(f_i \bigodot R \bigodot O \bigodot [\lambda,\ldots,\lambda])$
    \item Draw state sequence $z$, $z_{t+1}|z_{t} \sim \pi_i$
    \item Draw observations $y \sim Ber()\times MN() \times \exp(\lambda)$
  \end{itemize}
\end{enumerate}


\section{MCMC Sampler}
We need the following sampling routines:
\subsection{Sample $\theta,\eta|z,R,O$}
Given states and transition probabilities, $\theta$ and $\eta$ are conditionally independent. We sample them seperatly. Since all priors are conjugate, the sampling is easy.
\subsection{Sample $F|\theta,R,O,\eta$}
In order to sample shared features we use Metropolis–Hastings sampler by using IBP priors. One tricky part is computing observation likelihoods. For HMM case, it is basically forward-backward algorithm. For CTMC, ?.
\subsection{Sample $z|F,\theta,R,O,\eta$}
Sampling sequence of states given state transitions and observations can easily be accomplished via dynamic programming in HMM case. However; if we use CTMC,  designing an exact sampler is hard. Indeed, there is a way by using Bayes rule and impoartance sampling.
\subsection{Sample $c,\alpha|F$}
Classical HBP procedure by Fox et al\cite{fox}.
\subsection{Sample $\lambda,R,O|F,z,\eta,\theta$}
In order to sample $R,O$, we basically compute the sufficient statistics (current co-occurence and orders) and re-sample Bernouli random variables. For $\lambda$, ?.
%\subsection{Sample $F$ from $IBP$}

\section{Next Week(s) Plan}
\begin{itemize}
  \item Deriving full sampler by going into the details of the CTMC (1 Week).
  \item Artificial data generation (already completed)
  \item Implementing and debugging the algorithm on artificial data (1 Week).
\end{itemize}
\end{document}
